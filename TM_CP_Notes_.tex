
%Dades estil capçalera 
%\pagestyle{fancy}
%\rhead{\textbf{ CDVO  \\ 2n. Grau en  Matemàtiques}}
%\lhead{\textbf{Nom: }Marc G******* Ricardo \\ \textbf{NIU: }*****471}
%Perquè capiguen les coses a la capçalera 
%\setlength{\headheight}{25pt}

%%%%%%%%%%%%%%%%%%%%%%%%%%%%%%%%%%%%%%%%%%%%%%%%%%%%%%%%%%%%%%%%%%%%
%%%                                                              %%%
%%%            %El més bàsic i alguns paquest%                   %%%
%%%                                                              %%%
%%%%%%%%%%%%%%%%%%%%%%%%%%%%%%%%%%%%%%%%%%%%%%%%%%%%%%%%%%%%%%%%%%%%

%Declarem la classe de document i dimensions
\documentclass[10pt,twocolumn]{article}

%Paquets d'idioma i codificació de caràcters
\usepackage[utf8]{inputenc}
\usepackage[T1]{fontenc}
\usepackage[catalan]{babel}

%Paquets d'escriptura matemàtica
\usepackage{amsmath,amsfonts,amssymb,amsthm}

%Definicions,exemples,observacions...
\newtheorem{def:}{{\color{green}$\blacksquare$} Definició}
\newtheorem{ex:}{{\color{blue}$\blacksquare$} Exemple}
\newtheorem{obs:}{{\color{olive}$\blacksquare$} Observació}
\newtheorem{prop:}{{\color{orange} $\blacksquare$} Proposició}
\newtheorem{th:}{{\color{red}$\blacksquare$} Teorema}
\newtheorem{col:}{{\color{green}$\blacksquare$} Corol·lari}
\newtheorem{not:}{{\color{black}$\blacksquare$} Notació} 

%Modificacions i noves comandes
\newcommand{\C}{\ensuremath{\mathbb{C}}}
\newcommand{\R}{\ensuremath{\mathbb{R}}}
\newcommand{\Q}{\ensuremath{\mathbb{Q}}}
\newcommand{\Z}{\ensuremath{\mathbb{Z}}}
\newcommand{\N}{\ensuremath{\mathbb{N}}}
\newcommand{\B}{\ensuremath{\mathcal{B}}}
\newcommand{\esp}{\text{ }}

%Canviem el símbol de demostració final  de un quadrat en blanc a QED 
\renewcommand\qedsymbol{QED} 
%Serveix per recordar alguna cosa que s'ha de fer però en el moment fa mandrà
\newcommand{\nehh}[1]{\color{magenta} (* {#1} *) \normalcolor }

%%%%%%%%%%%%%%%%%%%%%%%%%%%%%%%%%%%%%%%%%%%%%%%%%%%%%%%%%%%%%%%%%%%%
%%%                                                              %%%
%%%                   %Altres paquets%                           %%%
%%%                                                              %%%
%%%%%%%%%%%%%%%%%%%%%%%%%%%%%%%%%%%%%%%%%%%%%%%%%%%%%%%%%%%%%%%%%%%%

%Serveix per tenir més colors disponibles
\usepackage[svgnames]{xcolor}

%Serveix per agrupar files i columnes
\usepackage{multirow} 

%Serveix per introduir imatges
\usepackage{graphicx}

%Serveix per ficar url's 
\usepackage{url}

%Serveix per que totes les referències que apareguin en el document pdf clicant-hi amb el ratolí, el visor pdf saltarà a la posició referenciada
\usepackage{hyperref} 

%Serveix per ampliar les possibilitats dels entorns de llistes
\usepackage{enumerate,paralist}

%Serveix per poder generat subfigures.
\usepackage{subfigure} 

%Paquets per a dibuixos amb GeoGebra
\usepackage{pstricks-add} 

%Serveix taxar coses bé com per exemple $\longrightarrow$ i $\exists$
\usepackage{centernot} 

%Serveix per ficar colors a les taules
\usepackage{colortbl} 

%Serveix per fer el estil de capselera
\usepackage{fancyhdr}

%Serveix per fotre coses a dalt i a baix
\usepackage{stackrel} 

%Serveix per afegir poemes \begin{poem}
\usepackage{poemscol} 

% Serveix per ficar caixes
\usepackage{tcolorbox}

% Serveix per ficar lletres d'un alfafet 

%%%%%%%%%%%%%%%%%%%%%%%%%%%%%%%%%%%%%%%%%%%%%%%%%%%%%%%%%%%%%%%%%%%%
%%%                                                              %%%
%%%                   %Utilitats visuals%                        %%%
%%%                                                              %%%
%%%%%%%%%%%%%%%%%%%%%%%%%%%%%%%%%%%%%%%%%%%%%%%%%%%%%%%%%%%%%%%%%%%%

%Dades pael títol del nostre document
\title{\emph{\color{redviolet!75!black} Lorem ipsum dolor sit amet.  Pla Docent {(PD)}}}
\author{ Taller de modelització \\ 2n. Grau en  Matemàtiques \\ Universitat Autònoma de Barcelona \\ Equip 15 (Grup A)\\ }
\date{Versió: \today}

%Serveix per peus de pàgina i més
\pagestyle{fancy}
\fancyhf{}
\fancyhead[LE,RO]{Pla Docent}
\fancyhead[RE,LO]{\thepage}
\fancyfoot[CE,CO]{\leftmark}
\fancyfoot[LE,RO]{\thepage}
\renewcommand{\headrulewidth}{2pt}
\renewcommand{\footrulewidth}{1pt}

%Serveix perquè el nombre de secció sigui més bonic.
\def\thesection{\textbf{$\mathit{\arabic{section}}$}}
\setcounter{section}{-1}

%Serveix per lletra inicial més elegant i vanitosa 
\usepackage{erewhon}
\usepackage{lipsum}
\usepackage{lettrine}
\usepackage{GoudyIn}
\definecolor{redviolet}{RGB}{52,47,75}
\usepackage{xcolor} 
\renewcommand{\LettrineFontHook}{\color{redviolet}\GoudyInfamily{}}
\setcounter{DefaultLines}{3}%
\usepackage{listings}


%%%%%%%%%%%%%%%%%%%%%%%%%%%%%%%%%%%%%%5
%%%%%%%%%%%%%%%%%%%%%%%%%%%%%%%%%%%%%%5
%%%%%%%%%%%%%%%%%%%%%%%%%%%%%%%%%%%%%%5
%%%%%%%%%%%%%%%%%%%%%%%%%%%%%%%%%%%%%%5
%%%%%%%%%%%%%%%%%%%%%%%%%%%%%%%%%%%%%%5


\usepackage{tikz}
\usepackage{varwidth}
\usepackage{linegoal}
\usepackage[explicit]{titlesec}
\usepackage[margin=1.5cm]{geometry}
\usepackage[Lenny]{fncychap}
\ChNameVar{\fontsize{14}{16}\usefont{OT1}{phv}{m}{n}\selectfont}
\ChNumVar{\fontsize{60}{62}\usefont{OT1}{ptm}{m}{n}\selectfont}
\ChTitleVar{\Huge\bfseries\rm}
\ChRuleWidth{1pt}
\newcommand{\cod}[1]{{ \color{redviolet}\texttt{#1}}}
\newtcolorbox{box1}[1]{
	colback=gray!5!white,
	colframe=gray!75!black,
	title={\Large #1}
}
\newtcolorbox{box2}[1]{
	colback=gray!5!white,
	colframe=redviolet!75!black,
	title={\Large #1}
}
\usepackage[nameinlink, capitalise, noabbrev]{cleveref}
\begin{document}
\tableofcontents
\vspace{5mm}
%%%%%%%%%%%%%%%%%%%%%%%%%%%%%%%%%%%%%%%%%%%%%%%%%%%%%%%%%%%%%%%%%%
%%%%%%%%%%%%%%%%%%%%%%%%%%%%%%%%%%%%%%%%%%%%%%%%%%%%%%%%%%%%%%%%%%
%%                                                              %%
%%               Introducció, prefaci, abstact                  %%
%%                                                              %%
%%%%%%%%%%%%%%%%%%%%%%%%%%%%%%%%%%%%%%%%%%%%%%%%%%%%%%%%%%%%%%%%%%
%%%%%%%%%%%%%%%%%%%%%%%%%%%%%%%%%%%%%%%%%%%%%%%%%%%%%%%%%%%%%%%%%%
\begin{tcolorbox}[colframe=white,colback=redviolet!20,sharp corners=all,size=minimal,halign=center,valign=center]
	\section{\textit{Comentari per a la lectura}}
\end{tcolorbox}
\lettrine{D}onades les disperses ideàs que hem obtingut al llarg de les sessions de treball hem estructurat el treball de tal forma que cada secció té un cert grau \textit{d'independència} respecte les altres. Això ho podem veure en que les conclusions de cada secció ja queden explicites al finalitzar cadascuna i no després. Tot i poder fer-se una lectura permutada de les seccions recomanem el ordre facilitat. 
\\ 
\newpage
%%%%%%%%%%%%%%%%%%%%%%%%%%%%%%%%%%%%%%%%%%%%%%%%%%%%%%%%%%%%%%%%%%
%%%%%%%%%%%%%%%%%%%%%%%%%%%%%%%%%%%%%%%%%%%%%%%%%%%%%%%%%%%%%%%%%%
%%                                                              %%
%%                    Estudi del enunciat                       %%
%%                                                              %%
%%%%%%%%%%%%%%%%%%%%%%%%%%%%%%%%%%%%%%%%%%%%%%%%%%%%%%%%%%%%%%%%%%
%%%%%%%%%%%%%%%%%%%%%%%%%%%%%%%%%%%%%%%%%%%%%%%%%%%%%%%%%%%%%%%%%%
\begin{tcolorbox}[colframe=white,colback=redviolet!20,sharp corners=all,size=minimal,halign=center,valign=center]
\section{Anàlisis del problema}
\end{tcolorbox}
\lettrine{F}em una lectura del enunciat que descriu el problema plantejat. 

\vspace{5mm}

\begin{tcolorbox}[colback=black!1,title=\textbf{Enunciat del problema},coltitle=black,colbacktitle=black!10]
\textit{
Un departament d'una universitat té diferents tasques docents assignades, que s'han de repartir entre els seus professors. Actualment es distribueixen segons les hores de classe de cada tasca. Se suposa que el nombre d'hores mesura l'esforç associat a una tasca, però en la pràctica això no és prou realista, la qual cosa genera desequilibris. Es tracta de trobar un mètode més equilibrat per valorar les tasques docents, que tingui en compte la demanda per cada tasca per part dels diferents professors. Es podria expressar aquesta demanda a través d'una mena de subhasta. S'haurien de tenir en compte algunes restriccions, com per exemple, que tothom faci la mateixa quantitat de docència o la restricció que hi hagi a cada departament.
}
\end{tcolorbox}
Influenciats per la lectura de \cite{tah} indexem el enunciat de forma conceptual mitjançant color i nombres\footnote{Únicament com a eina visual, eliminant la seva possible ambigüitat amb nombres enters}. Alternem el color de la font per facilitar la lectura. 
\newline
\begin{tcolorbox}[colback=black!1,title=\textbf{Enunciat del problema},coltitle=black,colbacktitle=black!10]

\textit{{\color{cyan!60}$\blacksquare$}$^{(01)}${\color{black!80}Un departament d'una universitat té diferents tasques docents assignades, que s'han de repartir entre els seus professors.}}
 
\textit{{\color{blue!60}$\blacksquare$}$^{(02)}$ Actualment es distribueixen segons les hores de classe de cada tasca. Se suposa que el nombre d'hores mesura l'esforç associat a una tasca, però en la pràctica això no és prou realista, la qual cosa genera desequilibris.}

\textit{{\color{green!60}$\blacksquare$}$^{(03)}$ {\color{black!80}Es tracta de trobar un mètode més equilibrat per valorar les tasques docents, que tingui en compte la demanda per cada tasca per part dels diferents professors.}}
 
\textit{{\color{purple!60}$\blacksquare$}$^{(04)}$Es podria expressar aquesta demanda a través d'una mena de subhasta.}
 
\textit{{\color{violet!60}$\blacksquare$}$^{(05)}${\color{black!80}S'haurien de tenir en compte algunes restriccions, com per exemple, que tothom faci la mateixa quantitat de docència o la restricció que hi hagi a cada departament.}}

\end{tcolorbox}
Fem el anàlisis o interpretació per blocs, mantenint el mateix codi d'indexació. Les paraules en negreta únicament tenen la finalitat de recordar els tecnicismes de \textit{investigació operativa} que ens faciliten l'abstracció del problema plantejat per l'enunciat.
\begin{itemize}
\item[{ \color{cyan!60} \underline{\underline{\normalcolor (01)}}}] El problema abstracte consisteix en una tasca de repartiment o assignació. Concretament, els \textbf{objectes a repartir} són les tasques docents que han de ser repartides entre  el professorat, cada possible assignació s'anomenara \textbf{pla docent} o \textbf{solució} de forma anàloga en funció del context.\\
\item[{ \color{blue!60} \underline{\underline{\normalcolor (02)}}}] Acceptem que el \textit{Model actual} genera solucions \textbf{subòptimes} i això ho  justificarem amb els mateixos arguments del enunciat.\\
\item[{ \color{green!60} \underline{\underline{\normalcolor (03)}}}] És requereix { \color{green!60} \underline{\normalcolor mètode}} per { \color{green!60} \underline{\normalcolor valorar}} i { \color{green!60} \underline{\normalcolor repartir}} tasca docent en funció de la demanda del professorat. A més ha de poder aportar una solució millor \footnote{Equivalentment menys desequilibrada.}. 
\item[{ \color{purple!60} \underline{\underline{\normalcolor (04)}}}] Es proposa com a \textbf{alternativa}  desenvolupar un  { \color{purple!60} \underline{\normalcolor mètode basat en subhasta}}.
\item[{ \color{violet!60} \underline{\underline{\normalcolor (05)}}}] S'expressa anticipadament que el \textbf{model/mètode} ha de tenir \textbf{restriccions}  i s'expliciten dos de necessàries. El volum del treball ha de ser \textit{homogeni}\footnote{Entesa com la qualitat de: quantitat de docències semblants entre el professorat}. El mètode ha de contemplar la possibilitat de restriccions pròpies del departament. 
\end{itemize}
Analitzat i desglossat el enunciat, passem a concretar o tipificar els objectius del treball.
\begin{tcolorbox}[colback=black!1,title=\textbf{0bjectius},coltitle=black,colbacktitle=black!10]
	\begin{enumerate}
	\item Elaborar \textbf{Mètode} per generar plans docents universitaris{\color{cyan!80} $^{(01)}$}. Aquest estarà basat en un repartiment de les tasques docents multivariable (dependrà de més d'una \textbf{variable decidible}) per a mesurar l'esforç{\color{blue!80} $^{(02)}$}{\color{green!80} $^{(03)}$} de cada tasca. El mètode haurà d'evitar desequilibris tipificats{\color{blue!80} $^{(02)}$}(documentats o previsibles) i estarà subjecte a un volum de \textbf{restriccions}{\color{violet!60} $^{(05)}$} variables. El mètode serà estructurat en base a la demanada del professorat com és requereix.
	\item S'estudiara l'idea de fer un model  basat en les subhastes.{\color{purple!80} $^{(04)}$} 
	\item S'aplicara el mètode en un cas concret per tal de comprovar la seva viable implementació i competència. El cas concret serà el \textit{Departament de Matemàtiques} de la pròpia Universitat Autònoma de Barcelona.
	\end{enumerate}
\end{tcolorbox}
\vspace{5mm}
\newpage
%%%%%%%%%%%%%%%%%%%%%%%%%%%%%%%%%%%%%%%%%%%%%%%%%%%%%%%%%%%%%%%%%%
%%%%%%%%%%%%%%%%%%%%%%%%%%%%%%%%%%%%%%%%%%%%%%%%%%%%%%%%%%%%%%%%%%
%%                                                              %%
%%                    Estudi Model Actual                       %%
%%                                                              %%
%%%%%%%%%%%%%%%%%%%%%%%%%%%%%%%%%%%%%%%%%%%%%%%%%%%%%%%%%%%%%%%%%%
%%%%%%%%%%%%%%%%%%%%%%%%%%%%%%%%%%%%%%%%%%%%%%%%%%%%%%%%%%%%%%%%%%
\begin{tcolorbox}[colframe=white,colback=redviolet!20,sharp corners=all,size=minimal,halign=center,valign=center]
\section{Estudi del model Actual}
\end{tcolorbox}
\lettrine{A}questa secció és una aproximació intuïtiva al \textbf{model actual}. Repetidament fem i farem servir el nom de \textit{Model actual}  per referir-nos al model/mètode\footnote{Actualment, 2019.} que s'utilitza per repartir les tasques docents entre el professorat\footnote{En el Departament de Matemàtiques de l'universitat.}

Els objectius de la secció a  part d'obtenir una visió aproximada del \textit{cas particular} de major interès, tenen intenció d'extreure i prendre constància dels desequilibris i restriccions, de la secció anterior,({\color{blue!60}$\blacksquare$}$^{(02)}$) i ({\color{violet!60}$\blacksquare$}$^{(05)}$)  respectivament. 
\\

\subsection{Dades obtingudes de la Web del Departament}
Algunes de les dades del model actual les podem trobar a la \textit{pàgina web} del Departament \cite{webdep}.

\vspace{3mm}

\begin{tcolorbox}[colback=black!1,title=\textbf{Dades publiques a la web},coltitle=black,colbacktitle=black!10]
\textbf{Docència de Grau}\\
El Departament de Matemàtiques és responsable principal de tres graus:
	\begin{enumerate}
		\item Grau de Matemàtiques
		\item Grau d'Estadística Aplicada
		\item Grau de Matemàtica Computacional i Analítica de Dades
		\item A banda, el Departament té assignada docència de gran varietat d'assignatures de vint-i-sis titulacions diferents.
	\end{enumerate}
	\textbf{Estructura}\\
	El Departament està constituït per cinc unitats, que es corresponen amb les àrees de coneixement que té adscrites:
	\begin{enumerate}
		\item Àlgebra
		\item Anàlisi Matemàtica
		\item Estadística i Investigació Operativa
		\item Geometria i Topologia
		\item Matemàtica Aplicada
	\end{enumerate}
\end{tcolorbox}
	%\url{https://www.uab.cat/web/departament-de-matematiques\\-1194422425366.html}
	%\end{box2}
	%\begin{box1}{Model actual {\footnotesize (A.Ruiz)}}
\subsection{Dades facilitades pel Secretari del departament}\footnote{Actualment, càrrec ocupat pel Doctor Albert Ruiz Cirera.}
Donat que les dades publiques de la web són limitades i insuficients per tenir un primer contacte amb el \textit{problema}, s'ens ofereix la possibilitat de concretar una entrevista amb el actual secretari del departament. D'aquesta obtenim un nou esquema que resumeix prou bé la tasca a la qual ha de fer front el mètode objectiu del treball.
\vspace{3mm}

\begin{tcolorbox}[colback=black!1,title=\textbf{Dades del funcionament intern del \textit{model actual}},coltitle=black,colbacktitle=black!10]
	\textbf{Titulacions que demanen docència}
	\begin{itemize}
		\item Horari fixat
		\item Nombre d'alumnes fixat.
		\item Hores i tipologia fixada (Això vol dir: Teo, Semin, Probl).
	\end{itemize}
	En total unes 500 sol·licituds $\approx$ 150 assignatures, assignatura: (3h de teoria), (1 hora de problemes), (2 hores seminaris).
	\\
	\textbf{Professorat}
	\begin{itemize}
		\item 90 hores/any.
		\item $\vdots$
		\item 240 hores/any.
		\item (60 hores/any els estudiants de doctorat) $\approx$ \textit{soroll}.
	\end{itemize}
	Podem estimar que hi ha uns 100 professors.\\
	\textbf{Algunes de les normes que s'apliquen}
	\begin{itemize}
		\item Si un professor a fet una assignatura un any  té preferència per repetir-la.
		\item 3 anys com a màxim en assignatures dels 3 graus.
		\item 4 anys la resta d'aquestes.
		\item També hi ha altres càrrecs 
	\end{itemize}
	\textbf{Actualment s'intenta minimitzar}
	\begin{itemize}
		\item Dispersió: \# assignatures / professors. (\#1)
		\item Deutes personal (saldo de hores). (\#2)
	\end{itemize}
	\end{tcolorbox}

	\subsection{Altres notes del \textit{Model actual}}
	\begin{enumerate}
		\item Primer és fa una ronda de repartiment amb el professorat que té preferència per que repeteix una assignatura. Després és fa una segona ronda amb el que queda tot i que ha vegades hi ha canvis a ultim moment. En aquesta segona ronda és reparteix \textit{el que queda}, aquí  és on es fa un tractament més personal i més precís.
		\item En aquesta segona \textit{ronda} hi ha una llista de \textit{argument} que ajuden a seleccionar el millor repartiment, vegem-ne alguns exemples:
		\begin{enumerate}
			\item Que tots els professors puguin fer almenys alguna hora de teoria (en general compten més i són més valorades).
			\item Suposant que hi ha una assignatura molt desitjada de teoria que compten més hores, probablement s'assignara a algú que per exemple tingui un saldo de hores negatiu abans que algú que tingui un saldo positiu (això és que faci més del que ha de fer). També pot passar que entre dos sol·licitats s'assigni a qui fagi quadrar més ; $S(p_1)=-30$ i saldo de professor dos $S(p_2)=-5$, llavor una assignatura que val 20 hores s'assignara preferentment al primer.
			\item Aquestes estratègies buscant minimitzar (\#1) i (\#2) i a més tenir el personal \textit{content} que és una cosa molt difícil de modelitzar, però també és important.
			\item S'intenta que els alumnes de doctorat puguin fer unes poques hores d'alguna matèria que estigui relacionada amb el seu treball.
		\end{enumerate} 
		\item Hi ha normes prou bones que no s'han escrit, pel que fa les assignatures de 3r i 4rt  també tenen un filtre:
		\begin{enumerate}
			\item  Cada subdepartament només pot fer assignatures del bloc que li toca i si en vol fer alguna de una \textit{branca de coneixement} que no li pertoca ha de ser convidat pel subdepartament responsable.
			\item És a dir cada subdepartament té un \textit{mercat d'assignatures associat} que s'han de repartir en una \textit{Reunió}. (Això dóna una bona idea, ja que així si més gent està implicada en el repartiment és poden considerar un model gran format de petits models \textit{més tractable} que interactuen \textit{poc} entre ells).
			\end{enumerate}
		\end{enumerate}
\subsection{Estat del model actual}
Des del propi Departament de matemàtiques se'ns han facilitat dades que ens permeten calcular el saldo dels professors. Les dades facilitades són:\footnote{Per consultar-les vegeu el document \cod{pla\_docent.ods.}}
\begin{itemize}
	\item Nombre d'hores que ha de fer cada professor del departament en un any.$\quad(p_i, \esp i \in I)$
	\item Nombre d'hores que fa realment un professor per a cada assignatura que se li ha assignat en el repartiment. $\quad (p'_i, \esp i \in I)$
\end{itemize}
El nostre objectiu és extreure del document tota la informació rellevant  mitjançant eines informàtiques, com la programació en llenguatge C, obtenir la diferencia entre les hores que hauria de fer un professor i les hores que realment fa, és a dir, $D_i:=p'_i-p_i$.

En cas de que un professor fes més hores, el programa ens tornarà un resultat positiu en hores $D_i>0$. En cas contrari $D_i<0$ , el programa ens tornarà un resultat negatiu indicant així que el professor no arriba al nombre d'hores que ha de fer. Aquest també suma totes les hores que s'estan fent extra i totes les que no s'arriben a fer i ens les mostra per pantalla i , a continuació, suma aquests dos valors per obtenir el saldo total.\footnote{Per veure el codi del programa vegeu el fitxer \cod{saldo.c.}}
Executem el programa i aquests són els resultats:
\begin{itemize}
	\item El $\sum$ de saldos positius és \texttt{1793} hores.	
	\item El $\sum$ de saldos negatius és \texttt{671.0} hores.	
	\item El $\sum$ de saldos positius i negatius  és \texttt{-1122} hores, que és un \texttt{\textbf{8.968825 \%}} del nombre d'hores total que haurien de fer tots els professors.
	\item La mitjana aritmètica d' assignatures que fa cada professor és de  $\approx 5$.
\end{itemize}
Es fa evident doncs que el model és propens a reduir el nombre d'hores que s'haurien de fer i observant els resultats professor a professor veiem que hi han diferencies de fins a 80 hores de saldo negatiu o casos on el saldo positiu ascendeix fins a 62.
\subsection{Conclusions}
		\begin{enumerate}
			\item El model actual funciona prou bé, fixa normes generals i fa un tractament diferent per cada cas, aconseguint més satisfacció en les assignacions. Per tant podem aprofitar molt del model actual. 
			\item Les particularitats \textit{bones} del model actual és que donen un tractament prou adaptable a la \textit{biodiversitat} del professorat.
			\item Els aspectes negatius del model no són clars llevat de dedicació que requereix, necessita un temps de dedicació considerable per fer quadrar totes les assignacions satisfactòriament .
			\item Les dades que ens ens seran facilitades seran dues llistes amb els \textit{items demandats} i una altra amb el \textit{nombre de professors de cada tipus} en funció de hores docents; amb \textit{noms} xifrats o anònims.
		\end{enumerate}
\subsection{Dades per extreure del \textit{model actual} i recopilar}
\begin{enumerate}
	\item \textbf{Variables per a mesurar l'esforç} 
	
	$(Element,\esp detalls,\esp  \textbf{G}eneral \esp o \esp \textbf{E}specific)$
	\begin{itemize}
		\item 
		\item $\cdots$
		\item 
	\end{itemize}
\item \textbf{Desequilibris tipificats}

	$(Element,\esp detalls,\esp  \textbf{G}eneral \esp o \esp \textbf{E}specific)$
\begin{itemize}
	\item 
	\item 
	\item 
\end{itemize}
\item \textbf{Restriccions}

$(Element,\esp detalls,\esp  \textbf{G}eneral \esp o \esp \textbf{E}specific)$
\begin{itemize}
	\item 
	\item 
	\item 
\end{itemize}
\end{enumerate}
Arribat aquí, després de 
\lipsum{10}
\\
\newpage
%%%%%%%%%%%%%%%%%%%%%%%%%%%%%%%%%%%%%%%%%%%%%%%%%%%%%%%%%%%%%%%%%%
%%%%%%%%%%%%%%%%%%%%%%%%%%%%%%%%%%%%%%%%%%%%%%%%%%%%%%%%%%%%%%%%%%
%%                                                              %%
%%                    Model 02 o "neeh"                         %%
%%                                                              %%
%%%%%%%%%%%%%%%%%%%%%%%%%%%%%%%%%%%%%%%%%%%%%%%%%%%%%%%%%%%%%%%%%%
\begin{tcolorbox}[colframe=white,colback=redviolet!20,sharp corners=all,size=minimal,halign=center,valign=center]
\section{Segon model o mètode \textit{optimització}}
\end{tcolorbox}
\lettrine{U}na altra forma d'abordar el problema es començar definint què és una \textbf{alternativa bona}, donant una mesura explícita de com de bona es una \textbf{alternativa}, ja que fins ara s'utilitzava una mesura no definida basada en el coneixement ímplicit del sistema que tenia el professor que realitzava l'assignació. Donat P el conjunt finit de professors i $o \in N$ el nombre d'assignatures que s'han de distribuïr, definim la mesura mencionada com una funció:
$$f: P^o \longrightarrow R$$
tal que com més baix sigui $f(a)$ (donat $a \in P^o$) millor serà l'alternativa $a$.

Un cop haguem trobat una funció que compleixi això, trobar una alternativa bona passarà a ser un problema d'optimització, el qual tractarem a la secció OPTIMITZACIÓ \nehh.
\\
\subsection{Definició de la funció objectiu}
Determinar la \textbf{funció objectiu} resulta difícil, ja que aquesta està lligada al criteri subjectiu de cada professor. Malgrat això, podem aplicar diverses \textbf{restriccions} sobre \textbf{alternatives} que estan prohibides pel sistema (com per exemple un professor assignat a dos classes que es realitzin simultàniament), de manera que ens aquests casos la funció donarà infinit. En els altres casos, en els que hem de determinar la qualitat d'una \textbf{alternativa factible} ens trobem amb l'ambigüitat que havíem comentat anteriorment, ja que per cada professor aquesta funció seria diferent i hem de balancejar els desitjos d'uns professors amb els dels altres, ja que en molts casos ens trobarem que una \textbf{alternativa} que és millor per a uns, és pitjor pels altres. Hi ha diverses possibles sortides a aquesta problemàtica:
\begin{enumerate}
	\item \textbf{Basar-nos en dades Històriques} fent un recull del \textit{historial}  de alternatives previes, juntament amb les dades base que es van utilitzar per crear-les, podem intentar trobar la funció ímplicita que es va utilitzar per crear les alternatives. És a dir, buscar, utlitzant aproximadors de funcions, una funció sobre les dades base tal que, al optimitzar-la sobre cada un dels conjunts de dades base de cada any, doni una alternativa el més semblant possible a la solució que es va utilitzar aquell any. 
	\item \textbf{Obtenció de dades mitjançant enquesta} preguntant als professors directament que valoren més i creant una funció de forma artesanal basant-nos en els resultats.
	\item \textbf{Funcions personalitzables} a nivell individual per cada professor. A partir d'aquestes funcions individuals podem crear una funció global que estigui formada per un sumatori d'aquestes funcions tal que al optimitzar aquesta funció global estarem optimitzant les funcions individuals en conjunt.
\end{enumerate} 

\subsection{Optimització}
Optimitzar aquesta funció resulta un problema molt difícil, ja que tenim unes $\approx 100^{500}$ \footnote{Altrament $ \approx 10^{1000}$}
Inicialment vam utilitzar \textit{Simulated Annealing} amb control de mínims locals per tal d'optimitzar 
----
\\
\newpage
%%%%%%%%%%%%%%%%%%%%%%%%%%%%%%%%%%%%%%%%%%%%%%%%%%%%%%%%%%%%%%%%%%
%%%%%%%%%%%%%%%%%%%%%%%%%%%%%%%%%%%%%%%%%%%%%%%%%%%%%%%%%%%%%%%%%%
%%                                                              %%
%%        Mdeel d'assignacions per subhastes                    %%
%%                                                              %%
%%%%%%%%%%%%%%%%%%%%%%%%%%%%%%%%%%%%%%%%%%%%%%%%%%%%%%%%%%%%%%%%%%
\begin{tcolorbox}[colframe=white,colback=redviolet!20,sharp corners=all,size=minimal,halign=center,valign=center]
	\section{Model d'assignacions  per subhastes  }
\end{tcolorbox}
\lettrine{E}n aquesta secció abordarem la problemàtica d'assignacions mitjançant l'estructura de subhasta, pensant, com en les seccions anteriors, en la seva aplicació final. La intenció  és construir un mètode sobre l'estructura de subhasta que permeti  donar solució al problema dels repartiments.
\subsection{Model Kiwis}
\subsection{Repartiment kiwis}
Per a tal de fer més justa la repartició d'assignatures i de premiar l'esforç del professorat i la qualitat de la seva feina, proposem un model de subhastes on la moneda de canvi, els kiwis, siguin repartits entre els pujadors de la següent manera:
Cada professor parteix de 10 Kiwis.
Si el professor té acumulat un saldo positiu (ha fet més hores del que li pertoquen en els ultims 3 anys) se li sumen 5 Kiwis per cada 10 hores que tingui en el saldo : 
\begin{itemize}
	\item 5 Kiwis per un saldo de 0 a 10h
	\item 10 Kiwis per un saldo de 11 a 20h
	\item  Etc.
\end{itemize}

A partir del segon any de l'instauració d'aquest model, si el professor va haver de impartir $x$ matèries que no va triar (assignades mitjançant la funció després de fer totes les subhastes) se li sumaran $x$ Kiwis. A aquells professors que vagin aconseguir més d'un 4.5 de mitjana en les enquestes realitzades per l'alumnat l'any anterior,  se li sumaran 30 kiwis. 

\subsection{Funcionament subhasta}

Realitzarem una subhasta per blocs de matèries, cada professor omplirà les caselles de les matèries per les que vulgui pujar d'un pdf amb el valor en kiwis que està disposat  a pagar. El preu inicial de cada matèria és de 0 Kiwis, la puja mínima és de 1 Kiwi, i a partir d'aquí es pot augmentar el preu de 0.5 Kiwis en 0.5 Kiwis (és a dir, les pujes vàlides per a una assignatura són: 1K, 1.5K, 2K, 2.5K, 3K,3.5K, etc.)

Un programa recullirà els resultats i otorgarà cada matèria al major pujador. En cas d'empat, es quedarà la matèria aquell professor qui partís de més kiwis al començament. En cas de tornar a haver empat, la assignatura quedarà lliure amb el valor obtingut com a valor inicial per a la següent subhasta.

D'aquesta manera es realitzaran tres subhastes. (En la última hi haurà la possibilitat de pujar amb Kiwis negatius: els pofessors poden oferir-se a fer una assignatura a canvi de rebre fins a 5 kiwis a canvi (pujant per -5 Kiwis)).


\subsubsection{Revalorització hores}

\begin{itemize}
	\item Suposem que una assignatura s'acaba assignant per un valor > 0 Kiwis (a la pràctica $\geq$1), i que aquesta matèria està valorada en h hores. Aleshores, el valor horari final de l'assignatura serà 
	$$ 
	h' = h - \frac{num. profs. que. han. pujat}{num. profs. que. han. pujat + 1}
	$$
	$$ - \frac{preu. en. Kiwis. pagat}{preu. Kiwis +1}
	$$
	
	Així doncs, el valor final de la matèria pot arribar a ser de fins a dues hores menys del valor inicial h.
	
	\item Ara suposem que una assignatura s'acaba assignant per un valor $\leq$ 0 Kiwis o que directament no s'assigna, i que està valorada en h hores. Aleshores, el valor final de l'assignatura serà: $$h'= h +1 $$
	(Contarà com una hora més).
\end{itemize}

\subsubsection{Assignacions finals}

Amb aquesta nova revalorització horària, es repartiran les matèries no assignades mitjançant el mètode de la funció entre els professors que tinguin de moment cobert un valor inferior a $\frac{2M}{3}$ hores, on M són les hores que ha de realitzar aquell professor.
\subsection{Aquí va una part de Talloc}
\subsection{Problemàtiques de les  subhastes}
Aquí tenim un resum de les problemàtiques que ha de soluciona'l el model de subhasta que considerem construir 
\begin{tcolorbox}[colback=black!1,title=\textbf{Llista problemàtiques subhastes},coltitle=black,colbacktitle=black!10]
\begin{enumerate}
	\item Les subhastes seran \textbf{seqüencials} o \textbf{simultànies}? Ja que les simultànies  poden implicar un \textbf{bloqueig de moneda} i les seqüencials poden implicar la inviabilitat de les \textbf{sol·licituds condicionals} i \nehh{no recordo}, almenys \textit{ a priori}.
	\item Com assignem les assignatures que no ha sol·licitat ningú?
	\item Les situacions de condicionals, com ara la puja de fer una assignatura condicionada a si  és compleix una condició.{\color{gray} Per exemple impartir amb un docent amb qui es té  \textit{bon tracte}, o bé tenir interès en un conjunt d'assignatures com a global i no per separat, és a dir, per conveniència horària} 
	\item Cas pràctic. Elegir assignatures per conveniència horària més  que per preferència de la matèria a impartir, sota la suposició de ser un criteri raonable i \textit{comú} en les dades històriques. {\color{gray} Per exemple voler impartir classe el menor nombre de dies per dificultats en el transport.}  
	\item Tot sistema que incentivi a mentir afavorira la especulació que farà poc \textbf{eficient} l'assignació final.
	\item Una assignació \textbf{optima} en tot cas haurà de  minimitzar la diferencia entre les \textbf{hores reals} i les \text{hores ideals}.
	\item Com desitgem una \textbf{revaloració} \footnote{Que pot ser la mateixa o no.} de cada matèria docent subhastada assumim que ha de ser necessari que la subhasta influeixi en el valor final de cada assignatura, en principi el valor en hores s'haurà de modificar.
	\item Situació de empat. {\color{gray} En cas de \textbf{subhasta simultània} .}
\end{enumerate}
\end{tcolorbox}
\subsection{Construcció primer model}
De forma constructiva, i aprofitant la subsecció prèvia, construirem un model que compleixi les condicions  de la llista anterior.

\subsection{Model nasi}
% Porque hitlur solo queria pintar cuadros
\begin{itemize}
	\item Subhastes multidimensionals en que es tenen en compte diverses coses per a decidir el guanyador de la subhasta
	\item Els professors poden fer pujes condicionades sobre els resultats d'altres subhastes
	\item Al finalitzar subhastes els seus resultats són propagats a altres subhastes passades on hi havia pujes condicionades sobre el resultat de la subhasta actual -> Subhastes passades actualitzen el seu resultat i el propaguen.
	\item Realitzem subhastes -> Eliminem assignacions que sobrepassin hores -> Realitzem subhastes -> ...
\end{itemize}

Algoritme
\begin{enumerate}
	\item Els professors entren totes les seves puges (que poden estar condicionades sobre altres pujes)
	\item Executar totes les subhastes una rer l'altra. En cas de trobar un apuja condicionada sobre una subhasta que no s'ha executat es considera la puja com a bona i s'executa la subhasta com si fos una puja normal, depsrés es fa una anotació marcant que aquesta puja depén de els resultats futurs d'una altra subhasta futura. Quan l'altra subhasta s'executi, al obtenir els resultats es revisita la subhasta depenent i, en cas que la puja condicional ja no sigui vàlida, aquesta subhasta és tornar a executar ignorant la puja invalidada. Aquestes invalidacions es propaguen sobre totes les subhastes depenents.
	\item Buscar els professors als que els hi ha sigut assignat assignatures que superen el nombre d'hores que haurien de realitzar, desassignar-lis suficients assignatures com per a que el nombre d'hores totals de les assignatures assignades a ells sigui semblant al nombre d'hores que han de fer. A continuació, aquests professors, les seves pujes i les assignatures que encara tenen assignades són eliminats del sistema.
	\item Si en el pas anterior s'han eliminat professors del sistema tornar al pas 2. En cas contrari, ja hem acabat i l'assignació final ja està feta (queda assignar les asisgnatures que no ha volgut ningú i vigilar que no hi hagi professors que hagin quedat amb molt poques assignatures assignades)
\end{enumerate}
ñ
\\
\newpage
%%%%%%%%%%%%%%%%%%%%%%%%%%%%%%%%%%%%%%%%%%%%%%%%%%%%%%%%%%%%%%%%%%
%%%%%%%%%%%%%%%%%%%%%%%%%%%%%%%%%%%%%%%%%%%%%%%%%%%%%%%%%%%%%%%%%%
%%                                                              %%
%%                    Criteris o dades recopilació              %%
%%                                                              %%
%%%%%%%%%%%%%%%%%%%%%%%%%%%%%%%%%%%%%%%%%%%%%%%%%%%%%%%%%%%%%%%%%%
\section*{Apèndix}
\begin{enumerate}
	\item \textbf{Paràmetres a optimitzar } (com a criteris) 
	\begin{itemize}
		\item La diferencia entre el nombre d'hores que \textit{hauria de fer} un professor i les hores que \textit{finalment farà} ha de ser el més petita possible. {\color{gray} Considerem que és l'estat més ideal.} \textbf{(minimitzar)} 
		\item La quantitat d'assignatures assignades a cada professors ha de ser baixa. 
		{\color{gray}  Per evitar el cas de que part del professorat tingui un nombre elevat de matèries a impartir, malgrat contin poc cadascuna ja que puntualment és pot tenir un sobrevolem de treball.} \textbf{(minimitzar)}
		\item L'assignació de les assignatures entre el professorat s'hauria d'acostar el màxim possible a les sol·licituds particulars demandades. {\color{gray}  L'elecció ja és justificació prou per argumentar que és una \textbf{alternativa} desitjable.} \textbf{(maximitzar)}
		\item 
%{\color{gray} El motiu d'aquesta mesura és incentivar al professorat a invertir (opcionalment) en l'elaboració de material docent que malgrat el primer any de docencia li comporti un mayor esforç a la llarga sigui compensat, a més de millorar la docencia és clar.}
	\end{itemize}
    \item \textbf{Restriccions} (com a criteris)
    \begin{itemize}
    	\item No pot haver solapament d'hores entre les assignatures que ha d'impartir un mateix docent . {\color{gray} En essència això evita entre d'altres coses que un mateix docent hages d'estar a dos llocs diferents al'hora .} \textbf{(Restricció de existència)} 
    	\item El interval de temps entre el final d'una sessió de treball i el inici d'una altre ,impartides per un mateix professional docent, a diferents facultats ha de ser major o igual a 20 minuts. {\color{gray} Un model que no contemples la pèrdua de temps en efectuar desplaçaments físics difícilment serà eficient i probablement precari.} \textbf{(Restricció d'ordre)} 
    	\item Les assignatures dels graus:
    	\begin{enumerate} 
    		\item Grau de Matemàtiques.
    		\item Grau d'Estadística Aplicada.
    		\item Grau de Matemàtica Computacional i Analítica de Dades.
    	\end{enumerate}
      No podran ser impartides pel mateix docent durant més de tres anys seguits.
      {\color{gray} Aquesta restricció té justificació per imposició de dades històriques i pràctiques.} \textbf{(Restricció d'ordre)} 
    	\item  . {\color{gray} .} \textbf{(minimitzar)} 
    	\item La resta d'assignatures dels vint-i-sis graus restants estarà limitada a quatre anys seguits de docència per un mateix docent. {\color{gray} Això per una banda implica un compromís de cara a la matèria impartida per cada docent, i per altra banda garanteix una estabilitat en el temari impartit de cara als estudiants.} \textbf{(Restricció d'ordre)} 
    	\item  Les matèries  dels cursos tercer i quart impartides als tres graus propis del departament \footnote{Grau de Matemàtiques, grau d'Estadística Aplicada, grau de Matemàtica Computacional i Analítica de Dades}. Sols podran ser impartides per docents que pertanyin a la unitat/subdepartament més proper a la matèria.   {\color{gray} La justificació és prou raonable, donada la actual \textit{especificació} inherent a la ciència en la actualitat. Un exemple seria que l'assignatura de Topologia sols pot ser impartida per docents de \textit{l'unitat} o \textit{subdepartament} de Geometria i Topologia \footnote{Veure secció de  \textit{model actual}.} .} \textbf{(Restricció d'ordre)} 
    \end{itemize}																																					
   
\end{enumerate}
Dogmes de criteris
%%%%%%%%%%%%%%%%%%%%%%%%%%%%%%%%%%%%%%%%%%%%%%%%%%%%%%%%%%%%%%%%%%
%%%%%%%%%%%%%%%%%%%%%%%%%%%%%%%%%%%%%%%%%%%%%%%%%%%%%%%%%%%%%%%%%%
%%%%%%%%%%%%%%%%%%%%%%%%%%%%%%%%%%%%%%%%%%%%%%%%%%%%%%%%%%%%%%%%%%
%%                                                              %%
%%                    Bibliography                              %%
%%                                                              %%
%%%%%%%%%%%%%%%%%%%%%%%%%%%%%%%%%%%%%%%%%%%%%%%%%%%%%%%%%%%%%%%%%%
%%%%%%%%%%%%%%%%%%%%%%%%%%%%%%%%%%%%%%%%%%%%%%%%%%%%%%%%%%%%%%%%%%
\footnotesize
\bibliographystyle{ieeetr}
\bibliography{Bib.bib}
\end{document}
%https://youtu.be/mANIqat5IC8
